
\documentclass{article}
\usepackage[intlimits]{amsmath}
\usepackage{amssymb}
\usepackage{amsfonts,amstext,amsthm}
\usepackage{paralist}        % {inparaenum} environment
\usepackage{mathtools}       % {dcases} environment
\usepackage{wasysym}         % \clock
\usepackage[normalem]{ulem}  % \sout
\usepackage[usenames,dvipsnames]{xcolor} % Named colors
\usepackage{hyperref}
\usepackage{booktabs}
\usepackage[margin=2.5cm]{geometry}

\usepackage{tikz}
\usetikzlibrary{arrows.meta}
\usetikzlibrary{decorations.pathmorphing}
\usetikzlibrary{patterns}

% ~~ Styling ~~
\renewcommand{\geq}{\geqslant}
\renewcommand{\leq}{\leqslant}

\newcommand{\ignore}[1]{}
\newcommand{\nolabel}[1]{}
\newcommand{\set}[1]{\left\{ #1 \right\}}
% Set with condition and automatically scaled braces: {... | ...}.
\newcommand{\cset}[3][:]{\left\{#2 \,#1\, #3\right\}}

\renewcommand{\Pr}{{\sf P}}           % Probability measure.
\DeclareMathOperator{\EV}{{\sf E}}    % Expected value.
\DeclareMathOperator{\Var}{{\sf Var}} % Variance.
\DeclareMathOperator{\Cov}{{\sf Cov}} % Covariance.
\DeclareMathOperator{\SE}{{\sf SE}}   % Standard error.
\DeclareMathOperator{\Hyp}{\mathcal{H}}
\DeclareMathOperator{\DNormal}{\mathcal{N}} % Normal distribution.

% ~~ Linear Algebra ~~
\renewcommand{\vec}[1]{\boldsymbol{#1}}
\newcommand{\One}{\mathchoice{\rm 1\mskip-4.2mu l}{\rm 1\mskip-4.2mu l}{\rm 1\mskip-4.6mu l}{\rm 1\mskip-5.2mu l}}
\newenvironment{algorithm}[1][]{\paragraph*{Algorithm#1.}}{\vspace{1ex}}

% ~~ Paper-specific ~~
\newcommand{\hlambda}{\hat{\lambda}}
\newcommand{\hmu}{\hat{\mu}}
\newcommand{\htau}{\hat{\tau}}
\newcommand{\ESS}{\mathrm{ESS}}
\newcommand{\PFA}{\mathrm{PFA}}
\newcommand{\PMS}{\mathrm{PMS}}
\newcommand{\PrFA}{\Pr _\mathrm{FA}}
\newcommand{\PrMS}{\Pr _\mathrm{MS}}

% Styling.
\hypersetup{
    colorlinks=true,%
    bookmarksnumbered=true,%
    bookmarksopen=true,%
    citecolor=blue,%
    urlcolor=blue,%
    unicode=true,           % enable unicode encoded PDF strings
    breaklinks=true         % allow links to break over lines by making
                            % links over multiple lines into PDF
                            % links to the same target
}

\begin{document}


\section*{\texttt{aftermath::sequential}}

\paragraph*{\texttt{parallel\_stopping\_time} motivation.}
Consider a rule $(\tau, d)$ that decides between two hypotheses $\Hyp_0$ and $\Hyp_1$, where $\tau = \min\set{t_0, t_1}$,
\[
    t_0 = \inf\cset{n \geq 1}{R_n \geq a},
    \qquad
    t_1 = \inf\cset{n \geq 1}{S_n \geq b},
\]
and $d = j$ if $\tau = t_j$, $j = 0, 1$.
When both thresholds are crossed at once, i.e., $\tau = t_0 = t_1$, we consider the decision erroneous.

For efficiency reasons, rather than perform simulations for a given pair of thresholds $(a, b)$, we construct a grid of thresholds $(a_i, b_j)$ with $a_1 < a_2 < \cdots < a_m$ and $b_1 < b_2 < \cdots < b_n$.
We will call the thresholds $a$, $b$, null and alternative, respectively.
\begin{center}
    \begin{tikzpicture}[
            arrow/.style={->, >=Latex},
            box/.style={inner sep=0pt, minimum width=80pt, minimum height=60pt}
        ]
        \draw[arrow]  (-2,+1.4) -- (2,+1.4); \node at (2.7,+1.4) {$\ensuremath{b, \text{alt}}$};
        \draw[dashed] (-2,0) -- (-0.8,0); \node at (-2.2,0) {$\ensuremath{i}$};
        \draw[arrow]  (-1.8,1.6) -- (-1.8,-1.6); \node at (-1.8,-1.9) {$\ensuremath{a, \text{null}}$};
        \draw[dashed] (0,1.6) -- (0,0.3); \node at (0,1.9) {$\ensuremath{j}$};
        \node[draw, rectangle, box] at (0, 0) {$\ensuremath{(a_i, b_j)}$};
    \end{tikzpicture}
\end{center}

The stopping time $\tau$ will run as long as at least one of the threshold pairs in the matrix has not been crossed, or equivalently as long as $(a_m, b_n)$ has not been crossed. For each pair of thresholds, we record the observed value of $\tau$, and the decision made ($1$ if $d = 0$; $2$ if $d = 1$; $3$ if $\tau = t_0 = t_1$).

When the first observation is collected, the $\mathsf{\Gamma}$-shaped region (dotted on the figure below) in the corresponding matrices will be filled.
With each next step the decision will have been made in a $\mathsf{\Gamma}$-shaped region.
Let $i_k$ and $i_{k + 1}$ denote the indices of the first uncrossed null threshold at steps $k$ and $k + 1$, respectively.
Let $j_k$ and $j_{k + 1}$ denote the corresponding indices of the first uncrossed alternative thresholds.
Then the decisions associated with $\tau = k + 1$ will have the following form.
\begin{center}
    \newcommand{\scaled}[2]{(\xscale * #1, \yscale * #2)}
    \newcommand{\halfway}[4]{(\xscale * 0.5 * #1 + \xscale * 0.5 * #2, \yscale * 0.5 * #3 + \yscale * 0.5 * #4)}
    \begin{tikzpicture}[
            old/.style={pattern=dots, pattern color=gray},
            null/.style={pattern=north west lines, pattern color=gray},
            alt/.style={pattern=north east lines, pattern color=gray}]
        \pgfmathsetmacro{\xscale}{0.8}
        \pgfmathsetmacro{\yscale}{0.8}
        \pgfmathsetmacro{\delta}{0.05}
        \pgfmathsetmacro{\a}{2.0}
        \pgfmathsetmacro{\b}{4.0}
        \pgfmathsetmacro{\c}{8.0}
        \pgfmathsetmacro{\x}{-1.0}
        \pgfmathsetmacro{\y}{-2.0}
        \pgfmathsetmacro{\z}{-4.0}
        % ~~ Old stuff ~~
        \draw \scaled{0 - \delta}{0 + \delta} rectangle \scaled{\c + \delta}{\z - \delta};
        \draw[old] \scaled{0 + \delta}{0 - \delta} --
              \scaled{\c - \delta}{0 - \delta} --
              \scaled{\c - \delta}{\x + \delta} --
              \scaled{\a - \delta}{\x + \delta} --
              \scaled{\a - \delta}{\z + \delta} --
              \scaled{0 + \delta}{\z + \delta} -- cycle;
        % ~~ Indices ~~
        \draw \scaled{\a}{0 + \delta} node[above] {$\ensuremath{j_{k}}$};
        \draw \scaled{\b}{0 + \delta} node[above] {$\ensuremath{j_{k + 1}}$};
        \draw[dashed] \scaled{\a}{0 + \delta} -- \scaled{\a}{\x + \delta};
        \draw[dashed] \scaled{\b}{0 + \delta} -- \scaled{\b}{\x + \delta};
        \draw \scaled{0 - \delta}{\x} node[left] {$\ensuremath{i_{k}}$};
        \draw \scaled{0 - \delta}{\y} node[left] {$\ensuremath{i_{k + 1}}$};
        \draw[dashed] \scaled{0 - \delta}{\x} -- \scaled{\a - \delta}{\x};
        \draw[dashed] \scaled{0 - \delta}{\y} -- \scaled{\a - \delta}{\y};
        % ~~ New blocks ~~
        \draw \scaled{\a + \delta}{\x - \delta} rectangle \scaled{\b - \delta}{\y + \delta};
        \draw[null] \scaled{\b + \delta}{\x - \delta} rectangle \scaled{\c - \delta}{\y + \delta};
        \draw[alt] \scaled{\a + \delta}{\y - \delta} rectangle \scaled{\b - \delta}{\z + \delta};
        % ~~ Labels ~~
        \node[fill = white] at \halfway{\a}{\b}{\x}{\y} {err};
        \node[fill = white] at \halfway{\a}{\b}{\y}{\z} {$\ensuremath{\Hyp_0}$};
        \node[fill = white] at \halfway{\b}{\c}{\x}{\y} {$\ensuremath{\Hyp_1}$};
    \end{tikzpicture}
\end{center}


\end{document}
