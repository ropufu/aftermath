
\documentclass{article}
\usepackage[intlimits]{amsmath}
\usepackage{amssymb}
\usepackage{amsfonts,amstext,amsthm}
\usepackage{paralist}        % For {inparaenum} environment.
\usepackage{mathtools}       % For dcases.
\usepackage[normalem]{ulem}  % For strikethrough.
\usepackage[usenames,dvipsnames]{xcolor} % For named colors.
\usepackage{hyperref}        % For clickable refs.
\usepackage{subcaption}      % Allows to include subfigures, floats, etc.
%\usepackage{enumerate}
\usepackage{MnSymbol}

\usepackage[authoryear,sort]{natbib}
\hypersetup{
    unicode=false,          % non-Latin characters in Acrobat’s bookmarks
    pdftoolbar=true,        % show Acrobat’s toolbar?
    pdfmenubar=true,        % show Acrobat’s menu?
    pdffitwindow=false,     % window fit to page when opened
    pdfstartview={FitH},    % fits the width of the page to the window
    pdfnewwindow=true,      % links in new window
    colorlinks=true,        % false: boxed links; true: colored links
    linkcolor=red,          % color of internal links (change box color with linkbordercolor)
    citecolor=BrickRed,     % color of links to bibliography
    filecolor=BurntOrange,  % color of file links
    urlcolor=blue           % color of external links
} % \hypersetup
\renewcommand{\geq}{\geqslant}
\renewcommand{\leq}{\leqslant}

\usepackage[margin=2.5cm]{geometry}

\usepackage{graphicx}
\graphicspath{{./gfx/}}
\DeclareGraphicsExtensions{.eps}
\usepackage{epstopdf}
\usepackage{tikz}
\usetikzlibrary{arrows}
\usetikzlibrary{patterns}

%\usepackage[nott]{inconsolata} % for C++ code

% C++ styles.
%\newcommand{\cppfont}{\fontfamily{fi4}\selectfont}
%\newcommand{\cppkey}[1]{{\color{blue}#1}}
%\newcommand{\cppclass}[1]{{\color{mint}#1}}
%\newcommand{\cppparam}[1]{{\color{gray}#1}}

% Math
\newcommand{\slfrac}[2]{\left. #1 \middle/ #2 \right.}

% ~~ Theorems and such ~~
\newtheorem{theorem}{Theorem}
\newtheorem{lemma}{Lemma}
\newtheorem{corollary}{Corollary}
\newtheorem{proposition}{Proposition}
\theoremstyle{definition} %% or \theoremstyle{remark} will produce roman text
\newtheorem{definition}{Definition}
\newtheorem{remark}{Remark}

\newtheorem{AUXalgorithm}{Algorithm}
\newenvironment{algorithm}[1]
  {\renewcommand\theAUXalgorithm{#1}\AUXalgorithm\begin{enumerate}\item[]}
  {\end{enumerate}\endAUXalgorithm}

% ~~ Math operators ~~
\DeclareMathOperator{\EV}{\mathbf{E}} % expected value
\DeclareMathOperator{\Var}{\mathbf{Var}} % variance
\DeclareMathOperator{\Cov}{\mathbf{Cov}} % covariance
\DeclareMathOperator{\Corr}{\mathbf{Corr}} % correlation
\DeclareMathOperator{\SD}{\mathbf{s.d.}} % standard deviation

% ~~ Distributions ~~
\DeclareMathOperator{\DNorm}{\mathcal{N}} % normal distribution
\DeclareMathOperator{\DLogNorm}{\ln \DNorm } % normal distribution
\DeclareMathOperator{\DExp}{\mathrm{Exp}} % exponential distribution
\DeclareMathOperator{\DPareto}{\mathrm{Pareto}} % exponential distribution
\DeclareMathOperator{\DBeta}{\mathrm{Be}} % exponential distribution
\DeclareMathOperator{\DUnif}{\mathrm{Uniform}} % exponential distribution
\newcommand{\scbeta}[1]{\DBeta_{\left(0, #1\right)}} % scaled beta distribution

% ~~ Misc ~~
\newcommand{\ignore}[1]{}
\newcommand{\nolabel}[1]{}
\newcommand{\cache}[1]{\fbox{$#1$}}
\newcommand{\ceil}[1]{\lceil{#1}\rceil}
\newcommand{\floor}[1]{\lfloor{#1}\rfloor}
\newcommand{\round}[1]{\lsem{#1}\rsem}
\newcommand{\Mode}{\theta}
\newcommand{\Reals}{\mathbb{R}}     % Real numbers.

\begin{document}

\section{Overview}

The aim of this work is two-fold: first, to give an overview of the Ziggurat algorithm \citet{Marsaglia+Tsang} for unimodal absolutely continuous distributions; second, to provide one implementation if the algorithm. One of the beauties of the Ziggurat algorithm lies in the fact that while almost all the time it is a very efficient version of the classical rejection method, it allows to handle densities with infinite support---assuming it is known how to sample from the tail (in the normal case see, e.g., \citet{Marsaglia:64}).

\section{Formal Setup}

In what follows, let $f$ denote the density function; $\Mode $ denote the mode of the distribution (argument where $f$ achieves its maximum); and $T$ denote its survival function:
\[
    T(x) = \int _x ^{\infty} f(t) \,dt.
\]
We will start with the monotone decreasing density case; the monotone increasing case can be handled similarly.

%\subsection{Unimodal density}

The idea behind the Ziggurat algorithm is to cover the density function with $n \geq 2$ horizontal layers of the same area, where all the layers except for the bottom one are rectangular.
%
More formally, we start with a partition of the vertical interval
\[
    0 = f_0 < f_1 < \cdots < f_{n - 1} < f_{n} = f(\Mode ).
\]
For $1 \leq k \leq n - 1$ let
\begin{align*}
    a_k = \inf \big\{ x \,:\, f(x) > f_k \big\}, \quad
    b_k = \sup \big\{ x \,:\, f(x) > f_k \big\};
\end{align*}
put $a_n = b_n = \Mode $, and define the layers by
\begin{align*}
    \begin{dcases}
        L_0 = \big\{ (x, y) \,:\, 0 \leq y \leq \min \{f_1, f(x)\} \big\}, \\
        L_k = [a_k, b_k] \times [f_k, f_{k + 1}], \quad 1 \leq k \leq n - 1.
    \end{dcases}
\end{align*}
This layering is illustrated in Figure~\ref{fig:ziggurat:3_layers}.

\begin{figure}[!ht]
    \centering
    \newcommand{\xfactor}{4.5}
    \newcommand{\yfactor}{8.5}
    \newcommand{\plotfun}[1]{\yfactor * #1 * exp(-#1 * #1)}
    \newcommand{\plotcoord}[2]{(\xfactor*#1,{\plotfun{#2}})}
    \newcommand{\plotMode}{0.7071}
    \newcommand{\plotZero}{0}
    \newcommand{\plotAone}{0.1536} % unscaled f_1 = 0.15
    \newcommand{\plotAtwo}{0.2687} % unscaled f_2 = 0.25
    \newcommand{\plotBtwo}{1.2770} % unscaled f_2 = 0.25
    \newcommand{\plotBone}{1.5223} % unscaled f_1 = 0.15
    \newcommand{\plotXfrom}{0.0}
    \newcommand{\plotXto}{1.8}
    \begin{tikzpicture}[scale=1.0, >=stealth',
            every node/.style={anchor=base},
            help lines/.style={dashed, thick},
            axis/.style={<->},
            important line/.style={thick},
            connection/.style={thick, dotted}]
        % ~~ Rectangular layers ~~
        \draw[pattern=north west lines, pattern color=gray] \plotcoord{\plotAtwo}{\plotMode} rectangle \plotcoord{\plotBtwo}{\plotBtwo}
            node[above right, yshift=4ex] {$\ensuremath{L_2}$};
        \draw[pattern=north east lines, pattern color=gray] \plotcoord{\plotAone}{\plotBtwo} rectangle \plotcoord{\plotBone}{\plotBone}
            node[above right, yshift=2ex] {$\ensuremath{L_1}$};
        % ~~ Bottom layer ~~
        \fill[pattern=dots, pattern color=gray] plot[domain=\plotXfrom:\plotXto] (\xfactor*\x,{min(\plotfun{\plotBone},\plotfun{\x})})--(\xfactor*\plotXto,0)--(\xfactor*\plotXfrom,0)--cycle;
        \draw \plotcoord{\plotXto}{\plotXto} node[above right, yshift=-2ex] {$\ensuremath{L_0}$};
        % ~~ Density ~~
        \draw[important line] plot[smooth,domain=\plotXfrom:\plotXto] (\xfactor*\x,{\plotfun{\x}}) node[right] {};

        \draw[connection]
            \plotcoord{\plotAone}{\plotAone}--(\xfactor*\plotAone,0.0)
            \plotcoord{\plotAtwo}{\plotAtwo}--(\xfactor*\plotAtwo,0.0)
            \plotcoord{\plotMode}{\plotMode}--(\xfactor*\plotMode,0.0)
            \plotcoord{\plotBone}{\plotBone}--(\xfactor*\plotBone,0.0)
            \plotcoord{\plotBtwo}{\plotBtwo}--(\xfactor*\plotBtwo,0.0);

        \draw (\xfactor*\plotAone,-0.4) node {$\ensuremath{a_1}$};
        \draw (\xfactor*\plotAtwo,-0.4) node {$\ensuremath{a_2}$};
        \draw (\xfactor*\plotMode,-0.4) node {$\ensuremath{a_3 = \Mode = b_3}$};
        \draw (\xfactor*\plotBtwo,-0.4) node {$\ensuremath{b_2}$};
        \draw (\xfactor*\plotBone,-0.4) node {$\ensuremath{b_1}$};

        \draw[connection] \plotcoord{\plotAtwo}{\plotMode}--\plotcoord{\plotZero}{\plotMode} node[left] {$\ensuremath{f_3 = f(\Mode )}$};
        \draw[connection] \plotcoord{\plotAone}{\plotBtwo}--\plotcoord{\plotZero}{\plotBtwo} node[left] {$\ensuremath{f_2}$};
        \draw[connection] \plotcoord{\plotAone}{\plotBone}--\plotcoord{\plotZero}{\plotBone} node[left] {$\ensuremath{f_1}$};
        \draw[connection] (\xfactor*\plotZero,0.0) node[left] {$\ensuremath{f_0 = 0}$};

        % ~~ Axes ~~
        \draw[->] (\xfactor*\plotZero,-0.2)--(\xfactor*\plotZero,4.3); % Vertical axis.
        \draw[->] (-0.1+\xfactor*\plotXfrom,0.0)--(0.4+\xfactor*\plotXto,0.0); % Horizontal axis.
    \end{tikzpicture}
    \caption{One-sided unimodal Ziggurat algorithm with $3$ layers.}
    \label{fig:ziggurat:3_layers}
\end{figure}
%
We want to choose $f_1, \cdots , f_{n - 1}$ in such a way as to make sure that the area of each layer is the same:
\[
    |L_0| = |L_1| = \cdots = |L_{n - 1}| = V.
\]
Thus, one arrives at a (non-linear) system of $n$ equations with $n$ unknowns:
\begin{align} \label{eq:zigg_system}
    \begin{dcases}
        V = (1 - T(a_1)) + (b_1 - a_1) \, f_1 + T(b_1) \\
        V = (b_k - a_k) \, \big( f_{k + 1} - f_k \big), \quad 1 \leq k \leq n - 1.
    \end{dcases}
\end{align}

\begin{proposition}
    System \eqref{eq:zigg_system} has a unique solution.
\end{proposition}

Having set up the basics, let us review the principles of the Ziggurat algorithm.
\begin{itemize}
    \item The first step is to select one layer (uniformly) at random.
    \item The second step is to generate a (uniform) random point inside the selected layer. If the point happens to be under the graph of the density function, we accept it. Otherwise, we start over from the first step.
\end{itemize}

To facilitate these steps, note that every layer has a rectangular subset that lies entirely under the density curve. Specifically, all
\[
    B_k = [a_{k + 1} , b_{k + 1}] \times [f_k, f_{k + 1}] \subseteq L_k, \quad 0 \leq k \leq n - 1,
\]
lie under the graph of $f$ (for the top layer the box $B_{n - 1}$ is trivial).
In light of this, the second step of the algorithm becomes substantially simpler if the point sampled from layer $L_k$ falls inside $B_k$. The probability of this happening, which we will refer to as the \emph{simple coverage probability}, is
\[
    p_k = \frac{|B_k|}{|L_k|}.
\]
It is convenient to write these probabilities in terms of the widths of the layers (the term ``width'' is accurate for all layers but the bottom one, in which case we interpret it as the width of a rectangle of the same height and area). Let
\[
    a_0 = a_1 - (1 - T(a_1)) / f_1
    \quad \text{and} \quad
    b_0 = b_1 + T(b_1) / f_1,
\]
and set
\begin{align} \label{eq:layer_width}
    \begin{dcases}
        %w_0 = V / f_1, \\
        w_k = b_k - a_k, \quad 0 \leq k \leq n - 1, \\
        w_n = 0.
    \end{dcases}
\end{align}
Then the simple coverage probabilities become
\begin{align} \label{eq:simple_coverage_probability}
    p_k &= \frac{w_{k + 1}}{w_{k}} \quad \text{for $0 \leq k \leq n - 1$}.
\end{align}
%
Furthermore, let $\alpha _k$ and $\beta _k$, $0 \leq k \leq n - 1$, denote the probabilities of landing to the left and right of $B_k$, respectively. Then
\begin{align} \label{eq:fallout_probability}
    \begin{dcases}
        \alpha _k = (a_{k + 1} - a_{k}) / w_k, \\
        \beta _k = (b_{k} - b_{k + 1}) / w_k,
    \end{dcases}
    \quad \text{for $0 \leq k \leq n - 1$}.
\end{align}
Note, that in the case of monotone densities, either all $\alpha _k = 0$ or all $\beta _k = 0$, $0 \leq k \leq n - 1$.
%
It is not difficult to see that
\begin{align*}
    \Pr \big(\text{simple coverage}\big) &= \frac{1}{n} \sum _{k = 0} ^{n - 1} \Pr \big(\text{simple coverage} \mid \text{layer $k$}\big)
        = \frac{p_0 + \cdots + p_{n - 1}}{n}, \\
    \Pr \big(\text{rejection}\big) &= \frac{1}{n} \sum _{k = 0} ^{n - 1} \Pr \big(\text{rejection} \mid \text{layer $k$}\big)
        = 1 - \frac{1}{n V}.
\end{align*}


\section{Ziggurat Algorithm}

The algorithm of \citet{Marsaglia+Tsang} and \citet{Doornik:05} for $n \geq 2$ layers is summarized below.

\begin{algorithm}{(Continuous generators)}
    \item \label{item:alg_cont:choose_box} Generate $K \sim \DUnif \{ 0, 1, \cdots , n - 1 \}$, the zero-based layer index.

    \item \nolabel{item:alg_cont:horizontal} Generate $U \sim \DUnif [0, 1)$, the relative ``horizontal position'' inside $L_K$.

    \item \nolabel{item:alg_cont:bottom_layer} If $K = 0$:
        \begin{enumerate}
            \item If $U < \cache{\alpha _0}$, accept a variate from the left tail.
            \item If $U \geq \cache{1 - \beta _0}$, accept a variate from the right tail.
            \item If $\cache{\alpha _0} \leq U < \cache{1 - \beta _0}$, accept $\cache{a_0} + U \cdot \cache{w_0}$, since the random point landed inside $B_0$.
        \end{enumerate}

    \item \nolabel{item:alg_cont:other_layers} If $K \neq 0$:
        \begin{enumerate}
            \item Let $X = \cache{a_K} + U \cdot \cache{w_K}$ be the abscissa of the random point inside $L_K$.
            \item If $\cache{\alpha _K} \leq U < \cache{1 - \beta _K}$, accept $X$, since the point landed inside $B_K$.
            \item Generate $W \sim \DUnif [0, 1)$, the relative vertical position inside $L_K$.
            \item If $\cache{f_K} + W \cdot \cache{(f_{K + 1} - f_{K})} < f(X)$, accept $X$, since the point landed under the graph of $f$.
            \item Return to step \ref{item:alg_cont:choose_box}.
        \end{enumerate}
\end{algorithm}
%
The ``boxed'' \fbox{quantities} in the algorithm above can be pre-computed to speed up calculations.

In practice one usually has to deal with discrete uniform integer generators. For simplicity of presentation, we will assume that the possible values are $\{ 0, 1, \cdots , M \}$; typically, $M$ would be a Mersenne number.
The previous algorithm can be adjusted to take advantage of integer arithmetic in the following way:
\begin{algorithm}{(Discrete uniform integer generators)}
    \item \label{item:alg_disc:choose_box} Generate $K \sim \DUnif \{ 0, 1, \cdots , n - 1 \}$, the zero-based layer index.

    \item \nolabel{item:alg_disc:horizontal} Generate $\hat{U} \sim \DUnif \{ 0, 1, \cdots , M \}$, the up-scaled relative ``horizontal position'' inside $L_K$.

    \item \nolabel{item:alg_disc:bottom_layer} If $K = 0$:
        \begin{enumerate}
            \item If $\hat{U} < \cache{\round{(M + 1) \alpha _0}}$, accept a variate from the left tail.
            \item If $\hat{U} \geq \cache{\round{(M + 1) (1 - \beta _0)}}$, accept a variate from the right tail.
            \item If $\cache{\round{(M + 1) \alpha _0}} \leq \hat{U} < \cache{\round{(M + 1) (1 - \beta _0)}}$, accept $\cache{a_0} + \hat{U} \cdot \cache{w_0  / (M + 1)}$, since the random point landed inside $B_0$.
        \end{enumerate}

    \item \nolabel{item:alg_disc:other_layers} If $K \neq 0$:
        \begin{enumerate}
            \item Let $X = \cache{a_K} + \hat{U} \cdot \cache{w_K / (M + 1)}$ be the abscissa of the random point inside $L_K$.
            \item If $\cache{\round{(M + 1) \alpha _K}} \leq \hat{U} < \cache{\round{(M + 1) (1 - \beta _K)}}$, accept $X$, since the point landed inside $B_K$.
            \item Generate $\hat{W} \sim \DUnif \{0, 1, \cdots , M \}$, the up-scaled relative vertical position inside $L_K$.
            \item If $\cache{f_K} + \hat{W} \cdot \cache{(f_{K + 1} - f_{K}) / (M + 1)} < f(X)$, accept $X$, since the point landed under the graph of $f$.
            \item Return to step \ref{item:alg_disc:choose_box}.
        \end{enumerate}
\end{algorithm}
%
As before, the ``boxed'' \fbox{quantities} in the discrete version of the algorithm can be pre-computed; moreover, some of them are now integer-valued. Note, that if we know in advance that $\alpha _0 = \cdots = \alpha _{n - 1} = 0$ or $\beta _0 = \cdots = \beta _{n - 1} = 0$, certain steps in the algorithm can be simplified (or skipped entirely).

\begin{remark}
    In most cases, we think it is reasonable to make the following assumptions.
    \begin{itemize}
        \item If $\alpha _j = 0$ for some $j$, then all $\alpha _k = 0$, $0 \leq k \leq n - 1$.
        \item If $\beta _j = 0$ for some $j$, then all $\beta _k = 0$, $0 \leq k \leq n - 1$.
    \end{itemize}
\end{remark}

\begin{remark}
    Also note, that only $\ceil{\log _2 (n)}$ bits are necessary to generate $K$; the remaining bits may be used, e.g., to implement vectorized Ziggurat versions, or store the random index for repeated iterations.
\end{remark}

\begin{remark}
    A couple of words on probability re-scaling, discrete generators, and rounding up or down.
    Suppose $M$ is a positive integer, $Y \sim \DUnif [0, 1)$, and $X = \floor{(M + 1) Y} \sim \DUnif \{ 0, 1, \cdots , M \}$. Our goal is to approximate the events $\{ Y < p \}$ or $\{ Y \geq p \}$ with events generated by $X$.
        
    To that end, first suppose $p = k / (M + 1)$ for some integer $k$, $0 \leq k \leq M + 1$. Then
    \[
        \{ Y < p \} = \{ X < k \}
        \qquad \text{and} \qquad
        \{ Y \geq p \} = \{ X \geq k \}.
    \]
    In general, if $0 \leq p \leq 1$, we want to round it toward the nearest multiple of $1 / (M + 1)$. Note, that in some cases $(M + 1)$ would result in an overflow, so we want to handle the case when $p \geq (M + 0.5) / (M + 1)$ separately.
\end{remark}

\bibliographystyle{plainnat}
\bibliography{ziggurat}


\end{document}
